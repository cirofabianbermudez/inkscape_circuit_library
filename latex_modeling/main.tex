\documentclass[10pt,letterpaper]{article}
\usepackage[utf8]{inputenc}
\usepackage[spanish]{babel}
\usepackage{amsmath}
\usepackage{amsfonts}
\usepackage{amssymb}
\usepackage{multicol}
\usepackage[left=2cm,right=2cm,top=2cm,bottom=2cm]{geometry}
\author{Ciro Fabián Bermudez Marquez}
\title{Libreria de circuitos}
\usepackage{circuitikz}[american]
\begin{document}

\begin{multicols}{3}
[
\section{Libreria}
Componentes comúnmente utilizados
]


%%--------------------------------------------------------------------------------------------
\begin{circuitikz}[american,scale=1, every node/.style={scale=1}]
 \draw
 (0,0) to[damper] ++(3,0)
 ;
\end{circuitikz}

%%--------------------------------------------------------------------------------------------
\begin{circuitikz}[american,scale=1, every node/.style={scale=1}]
 \draw
 (0,0) to[cV] ++(3,0)
 ;
\end{circuitikz}

%%--------------------------------------------------------------------------------------------
\begin{circuitikz}[american,scale=1, every node/.style={scale=1}]
 \draw
 (0,0) to[cI] ++(3,0)
 ;
\end{circuitikz}

%%--------------------------------------------------------------------------------------------
\begin{circuitikz}[american,scale=1, every node/.style={scale=1}]
 \draw
 (0,0) to[I] ++(3,0)
 ;
\end{circuitikz}

%%--------------------------------------------------------------------------------------------
\begin{circuitikz}[american,scale=1, every node/.style={scale=1}] %european
 \draw
 (0,0) to[V] ++(3,0)
 ;
\end{circuitikz}

%%--------------------------------------------------------------------------------------------
\begin{circuitikz}[american,scale=1, every node/.style={scale=1}]
 \draw
 (0,0) to[R] ++(3,0)
 ;
\end{circuitikz}

%%----------------------------------------------
\begin{circuitikz}[american,scale=1, every node/.style={scale=1}]
 \draw
 (0,0) to[C] ++(3,0)
 ;
\end{circuitikz}

%%----------------------------------------------
\begin{circuitikz}[american,scale=1, every node/.style={scale=1}]
 \draw
 (0,0) to[L] ++(3,0)
 ;
\end{circuitikz}

%%--------------------------------------------------------------------------------------------
\begin{circuitikz}[american,scale=1, every node/.style={scale=1}]
 \draw
 (0,0) to[D*] ++(3,0)
 ;
\end{circuitikz}

%%--------------------------------------------------------------------------------------------
\begin{circuitikz}[american,scale=1, every node/.style={scale=1}]
 \draw
 (0,0) to[sV] ++(3,0)
 ;
\end{circuitikz}

%%--------------------------------------------------------------------------------------------
\begin{circuitikz}[american,scale=1, every node/.style={scale=1}]
 \draw
 (0,0) to[short, o-o,v=$ $,i=$ $] ++(3,0)
 ;
\end{circuitikz}

%%--------------------------------------------------------------------------------------------
\begin{circuitikz}[american,scale=1, every node/.style={scale=1}]
 \draw
 (0,0) node [ground] {}
 ;
\end{circuitikz}

%%--------------------------------------------------------------------------------------------
\begin{circuitikz}[american,scale=1, every node/.style={scale=1}]
 \draw
 (0,0) node [sground] {}
 ;
\end{circuitikz}

%%--------------------------------------------------------------------------------------------
\begin{circuitikz}[american,scale=1, every node/.style={scale=1}]
 \draw
 (0,0) to[ammeter] ++(3,0)
 ;
\end{circuitikz}

%%--------------------------------------------------------------------------------------------
\begin{circuitikz}[american,scale=1, every node/.style={scale=1}]
 \draw
 (0,0) to[voltmeter] ++(3,0)
 ;
\end{circuitikz}

%%--------------------------------------------------------------------------------------------
\begin{circuitikz}[scale=1]\draw
(5,.5) node [op amp] (opamp) {}
;
\end{circuitikz}

%%--------------------------------------------------------------------------------------------
\begin{circuitikz}[scale=1]\draw
 (0,0) to[amp] ++(3,0)
;
\end{circuitikz}

%%--------------------------------------------------------------------------------------------
\begin{circuitikz}[scale=1]\draw
(5,.5) node[npn]{}
;
\end{circuitikz}

%%--------------------------------------------------------------------------------------------
\begin{circuitikz}[scale=1]\draw
(5,.5) node[pnp]{}
;
\end{circuitikz}

%%--------------------------------------------------------------------------------------------
\begin{circuitikz}[scale=1]\draw
(5,.5)node[nmos]{}
;
\end{circuitikz}

%%--------------------------------------------------------------------------------------------
\begin{circuitikz}[scale=1]\draw
(5,.5) node[pmos]{}
;
\end{circuitikz}

%%--------------------------------------------------------------------------------------------
\begin{circuitikz}[scale=1]\draw
 (0,0) to[battery] ++(3,0)
;
\end{circuitikz}

%%--------------------------------------------------------------------------------------------
\begin{circuitikz}[european,scale=1, every node/.style={scale=1}]
 \draw
 (0,0) to[R] ++(3,0)
 ;
\end{circuitikz}

%%-------------------------------------------------------------------------------------------- prueba
 \begin{circuitikz}[american,]
\ctikzset{tripoles/mos style/arrows}
\def\killdepth#1{{\raisebox{0pt}[\height][0pt]{#1}}}
\draw (0,0) node[nmos](Q1){};
\draw (Q1.center) node[right]{\killdepth{Q1}};
\end{circuitikz}

\end{multicols}
%%--------------------------------------------------------------------------------------------
%%--------------------------------------------------------------------------------------------
%%--------------------------------------------------------------------------------------------

\newpage
\vspace{0.5cm}
\begin{circuitikz}[american,scale=1, every node/.style={scale=1}]
 \draw
 (0,0) to[R] ++(3,0)
 ;
\end{circuitikz}

\vspace{0.5cm}
\begin{circuitikz}[american,scale=1, every node/.style={scale=1}]
 \draw
 (0,0) to[C] ++(3,0)
 ;
\end{circuitikz}

\vspace{0.5cm}
\begin{circuitikz}[american,scale=1, every node/.style={scale=1}]
 \draw
 (0,0) to[short, o-o,v=$ $,i=$ $] ++(3,0)
 ;
\end{circuitikz}

\vspace{0.5cm}
\begin{circuitikz}[american,scale=1, every node/.style={scale=1}]
 \draw
 (0,0) to[R,v=$n$,i=$n$] ++(3,0)
 ;
\end{circuitikz}

\vspace{0.5cm}
\begin{circuitikz}[american,scale=1, every node/.style={scale=1}]
 \draw
 (0,0) node [ground] {}
 ;
\end{circuitikz}

\vspace{0.5cm}
\begin{circuitikz}[american,scale=1, every node/.style={scale=1}]
 \draw
 (0,0) node [sground] {}
 ;
\end{circuitikz}

\vspace{0.5cm}
\begin{circuitikz}[american,scale=1, every node/.style={scale=1}]
 \draw
 (0,0) to[D*] ++(3,0)
 ;
\end{circuitikz}

\vspace{0.5cm}
\begin{circuitikz}[american,scale=1, every node/.style={scale=1}]
 \draw
 (0,0) to[ammeter] ++(3,0)
 ;
\end{circuitikz}

\vspace{0.5cm}
\begin{circuitikz}[american,scale=1, every node/.style={scale=1}]
 \draw
 (0,0) to[voltmeter] ++(3,0)
 ;
\end{circuitikz}

\vspace{0.5cm}
\begin{circuitikz}[american,scale=1, every node/.style={scale=1}]
 \draw
 (0,0) to[zzD*] ++(3,0)
 ;
\end{circuitikz}

\vspace{0.5cm}
\begin{circuitikz}[american,scale=1, every node/.style={scale=1}]
 \draw
 (0,0) to[sV] ++(3,0)
 ;
\end{circuitikz}


\begin{circuitikz}[scale=1]\draw
(5,.5) node [op amp] (opamp) {}
;\end{circuitikz}


\begin{circuitikz}[scale=1]\draw
(5,.5) node [op amp] (opamp) {}
;\end{circuitikz}


%%--------------------------------------------------------------------------------------------
\begin{circuitikz}[american,scale=1, every node/.style={scale=1}]
 \draw
 (0,0) to[R] ++(3,0)
 ;
\end{circuitikz}











\newpage
%%--------------------------------------------------------------------------------------------
\begin{circuitikz}[scale=1]\draw
(5,.5) node [ieeestd and port] (pepe) {}
;
\end{circuitikz}

%%--------------------------------------------------------------------------------------------
\begin{circuitikz}[scale=1]\draw
(5,.5) node [ieeestd nand port] (pepe) {}
;
\end{circuitikz}

%%--------------------------------------------------------------------------------------------
\begin{circuitikz}[scale=1]\draw
(5,.5) node [ieeestd or port] (pepe) {}
;
\end{circuitikz}


%%--------------------------------------------------------------------------------------------
\begin{circuitikz}[scale=1]\draw
(5,.5) node [ieeestd nor port] (pepe) {}
;
\end{circuitikz}


%%--------------------------------------------------------------------------------------------
\begin{circuitikz}[scale=1]\draw
(5,.5) node [ieeestd xor port] (pepe) {}
;
\end{circuitikz}


%%--------------------------------------------------------------------------------------------
\begin{circuitikz}[scale=1]\draw
(5,.5) node [ieeestd xnor port] (pepe) {}
;
\end{circuitikz}


%%--------------------------------------------------------------------------------------------
\begin{circuitikz}[scale=1]\draw
(5,.5) node [ieeestd buffer port] (pepe) {}
;
\end{circuitikz}



%%--------------------------------------------------------------------------------------------
\begin{circuitikz}[scale=1]\draw
(5,.5) node [ieeestd not port] (pepe) {}
;
\end{circuitikz}


%%--------------------------------------------------------------------------------------------
\begin{circuitikz}[scale=1]\draw
(5,.5) node [ieeestd and port, color=red, number inputs=4, circuitikz/ieeestd ports/height=3] (pepe) {}
;
\end{circuitikz}


%%--------------------------------------------------------------------------------------------
\begin{circuitikz}[scale=1]\draw
(5,.5) node [ieeestd xor port, number inputs=4] (pepe) {}
;
\end{circuitikz}




\newpage
%%--------------------------------------------------------------------------------------------
 \tikzset{flipflop AB/.style={flipflop,
flipflop def={t1=A, t3=B, t6=Q, t4={\ctikztextnot{Q}},
td=rst, nd=1, c2=1, n2=1, t2={\texttt{CLK}}},
}}

\begin{circuitikz}[scale=1]\draw
(5,.5) node[flipflop AB]{}
;
\end{circuitikz}

%%--------------------------------------------------------------------------------------------
\begin{circuitikz}[scale=1]\draw
(5,.5) node[flipflop D]{}
;
\end{circuitikz}

%%--------------------------------------------------------------------------------------------
\begin{circuitikz}[scale=1]\draw
(5,.5) node[ALU]{\rotatebox
{90}{\small \ttfamily ALU}}
;
\end{circuitikz}

%%--------------------------------------------------------------------------------------------
\tikzset{mux 4by2/.style={muxdemux,
muxdemux def={Lh=4, NL=2, Rh=3,
NB=1, w=2, square pins=1}}}

\begin{circuitikz}[scale=1]\draw
(5,.5) node[mux 4by2]{MUX}
;
\end{circuitikz}


%%--------------------------------------------------------------------------------------------
\begin{circuitikz}[scale=1]\draw
(5,.5) node[flipflop D]{DFF}
;
\end{circuitikz}





\end{document}